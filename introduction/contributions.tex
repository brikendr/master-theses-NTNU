\section{Contributions}
The primary contribution of this thesis is to provide an approach that effectively tackles the problem of named entity disambiguation which ultimately leads towards the enrichment of unstructured documents on the web with semantic content. More specifically, the approach will consist of a complete framework that will deal with extracting named entities, automatically linking them with knowledge bases and provide means of formulating the surrounding context of an entity so that human annotators can accurately distinguish between bad and good generated entity candidates. The framework serves as the basis on top of which a gamified system will be implemented. For the gamified system to have the desired outcome, namely intrinsically motivating users to play the game as well as retaining them in long-time periods, the game design will be based on theoretical foundations and basic psychological needs satisfaction. More specifically, Self-Determination Theory (\ac{sdt}) and its respective sub-theories will be explored and applied in our game design.

The second contribution of this work is a model which represents best practices on how to perform gemificaion on non-gaming contexts. We design a \ac{gwap} that truly engages players with its well-designed, task-oriented game elements that contribute a great deal to player intrinsic motivation and generation of qualitative annotations. With a \ac{gwap} designed and implemented on top of a microservice framework, we open up doors for further contributions to the research field. Small modifications or additional integration to the microservice framework, it will be possible to generate annotation data for other problems in the filed of NLP such as (potentially) language translation and speech recognition. Using this system, researchers and developers will be able to generate training data at minimal costs with data quality comparable to linguistic experts or trained annotators.

The complete implementation of the Named Entity Disambiguation Framework is open and available for everyone who is interested in utilizing it for similar or other research problems\footnote{AnnotateMe Framework \url{https://github.com/brikendr/AnnotateMeFramework}}. Additionally, the complete gameplay of Fastype (our gamified version of named entity disambiguation) is demonstrated in a video which is uploaded on Youtube and can be access through the following link\footnote{Fastype Gameplay \url{https://www.youtube.com/watch?v=FWJkHvHfj0U}}.