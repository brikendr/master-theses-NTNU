\section{Justification, motivation and benefits}

The content on web pages, news articles, blog posts and other internet data consists of hundreds and thousand mentions of named entities such as people, organizations, locations and other relevant concepts. These entities are often ambiguous, meaning that the same name can have many different meanings. Identifying these entities and linking them with a corresponding \ac{kb} that is part of the Link Open Data (\ac{lod}) initiative, results in several benefits for information processing and retrieval systems such as search engines. 

Despite the impressive enhancements of search engines in the last decade, information searching is still dependent on keyword-based searching which usually does not fully meet the users needs due to insufficient content meaning on the web documents \cite{5}. Since the techniques used by traditional search engines are based on straightforward matches of terms within unstructured text, understanding the context of the query created by the user is not taken into consideration \cite{18}. As a result users get frustrated by having to adjust their query terms to retrieve the desired results. The proposed solution to this problem is semantic search \cite{semantic_search}. Semantic search best operates when web documents contain semantic meta-data, which in turn allows the discovery of deeper meanings and relationships of specific query terms rather than relying on exact keyword matches \cite{18}. Previous research suggests that attaching semantic meta-data to unstructured web documents clearly improves precision in search engines with maximum recall \cite{18}. The problems addressed by our research study will contribute and have a significant impact towards the improvements of semantic search.

Since the performance of supervised approaches for \ac{ned} rely heavily on the availability of large training corpora, having a system that engages human users in an interactive and fun way can generate such corpora in a short period of time with minimal costs. According to Green et al. \cite{50}, knowledge captured by a specific annotated corpus is often not transferable to another task, even when it is the same NLP task but different language. This increases the importance of having a system which supports the generation of training data at minimal costs. 

Furthermore, research suggests that turning a task into a \ac{gwap} has shown to increase quality of results and higher user engagements, thanks to the users being stimulated by the playful component \cite{41}. Our motivation is to create a gamified framework model that is much in tune with the efforts of the Open Mind Initiative \footnote{Open Mind Initiative \url{http://wiki.p2pfoundation.net/Open_Mind_Initiative}} \cite{5} which focuses on the collection of data from internet users and using this data to train machine learning algorithms. 

Several other systems will benefit by having a reliable and accurate named entity disambiguation system. Information extraction, information retrieval, content analysis, question answering systems and knowledge base population are some of the applications where named entity disambiguation is considered as the initial step towards improving their accuracy and overall performance \cite{16}.

