\section{Related Work}
\label{game:relate-work}
The emergence of platforms which contribute to the concept of having a group of individuals commit to creating a collective solution that is far more powerful and robust than individual ones has drawn the curiosity of many industries during the recent years. This concept is referred to as Collaborative Resource Creation (\ac{crc}) and is being utilized by systems that inherently want to use the creative and powerful nature of human thinking as a source of solving different computationally complex problems. Wikipedia can be considered as the best known example of collaborative resource creation. Furthermore, Open Mind Common Sense (\ac{omcs}) movement demonstrated that Web collaboration can be used as a tool to create \ac{ai} resource. Games can also be considered as \ac{crc} system. \cite{44, 43}

Wikipedia and \ac{omcs}, as non-gamified systems, rely on peoples' altruism and interest on science in order to commit to contributing. Whereas games provide the feeling of being entertained, and as a result the solely rely on user intrinsic motivations. Von Ahn et al. \cite{vonahn} argues that the desire to be entertained is a much more powerful incentive than any other incentive technique. It has been estimated that more than 9 million person-hours are spent by people playing games on the WEB. Dedicating a small amount of those playing hours to contribute to the solution of complex computational problems will result in tremendous benefits. This is the reason why Games With A Purpose (\ac{gwap}) are being frequently used in many domains such as \ac{nlp} and Semantic Web. Gamifying a system for the purpose of solving or facilitating computationally complex problems can be unquestionably powerful when doing it right. An excellent successful example of such a system is Foldit \cite{53}. 

Games are also used in education, generally as serious games and as digital game-based learning. In this field, gamification is defined as the use of game-based mechanics, aesthetics and game thinking to engage people, motivate action, promote learning and solve problems \cite{47}. Seaborn et al. \cite{47} argues that gamification has been used as a means of collaborative resource creation by numerous studies which take advantage of the alleged motivational benefits that game design can provide. However, almost all of these attempts of \ac{gwap} lack empirical research and standard of practice for design and implementation \cite{47}. In our work, we have attempted to analyze and understand empirical theories such as \ac{sdt} and \ac{cet} which guided the implementation of a \ac{gwap} for named entity disambiguation which resulted in a collaborative resource creation system generating training data for supervised \ac{wsd} and \ac{ned} algorithms respectively. 

Seaborn et al. \cite{47} also reported statistics on the usage of gamificaion across domains ranging from sustainability to health and wellness to education. The findings reported by the study indicate that the fields in which gamification has been mostly applied are Education (35\%), health and wellness (13\%), online communities and social works (13\%), crowdsourcing (13\%) and sustainability (10\%). They also report that a large majority of applied gamification research did not mention or address any theoretical foundations \cite{47}.

Gamification has been used as a mechanism to solve various \ac{nlp} and semantic web problems. Kaboom \cite{41} is a gamified \ac{wsd} system that can be classified as a 2D video game in the style of Fruit Ninja game. In this game, players are asked to destroy pictures that are not related to a specific term or concept shown to them beforehand. This approach aims to disambiguate word senses by using pictures as sense descriptors. The pictures that are kept at the end of a game round are considered to be related senses for the ambiguous word. Senses for each ambiguous word in the game were collected manually by expert annotators. Unlike them, we use our implemented microservice framework for generating game data instead of manually creating them, which gives us a headstart in focusing more on the game design aspect. Similar to our findings, Jurgens et al. \cite{41} also reported that game-based annotation systems reduce the cost of producing equivalent resources via crowdsourcing at least by 73\% while providing similar quality of annotations. Using Kaboom \cite{41}, they also reached a 16.3\% improvement in accuracy over state-of-art \ac{wsd}.

Phrase Detective \cite{44} is another gamified system that was developed to annotate corpora for anaphora resolution. Anaphora resolution is a semantic task used for recognizing that a pronoun like "it" and the definite nominal "the town" refers to some entity as a proper name \cite{44}. They argue that a successful \ac{gwap} should make use of all available incentives, namely, personal, social and financial. The game interface should be easy to use, intuitive to learn and designed to engage ones intended player demographic. Furthermore, they emphasize validation as a strong and effective method for quality control. By collecting multiple judgments for each expression, the gamified system can provide quality control and collect useful linguistic data. We employed some of the proposed methods by Phrase Detective as they proved to be appropriate for the design of our game, Fastype. 

To help with the creation of named entities, Green et al. \cite{50} developed the \textit{Entity Discovery} game where players are asked to annotate sentences by marking all named entities found in it. The game is played in pairs and when real players are not online to be paired with, a BOT is used instead. In order to validate the recognized entities by the first game, they developed a second game called \textit{Name that Entity}. The second game was designed as a multiple choice game were the paired players had to choose the type of the recognized entity. The game accepts a specific type for the recognized entity only if the paired players agree upon the type. \cite{50} 
In contrast, we automated the entity recognition task using our framework and for the validation task we use multiple judgments and rely on agreement levels between players which proved to be very effective and assured high quality of annotations. 

Several research studies investigated individual game elements and their impact on intrinsic motivation and performance \cite{43,45,46}. Badges, leaderboards and performance graphs, as reported by Sailer et al. \cite{45}, positively affected competence and need satisfaction. The same game elements also seemed to contribute to an increase in perceived meaningfulness as it is known that these game elements can create meaning at the game level \cite{45}. Unsurprisingly, Mekler et al. \cite{46} found that pontsification (points, levels and laderboads) functioned as extrinsic incentives effective for promoting performance quantity. They used an image annotation game to determine the effect of these three most commonly employed game elements on needs satisfaction, intrinsic motivation and performance. Using a two-fold experimental study, the game element group performed better in terms of annotation quantity, whereas the quality remained the same. Against their expectations, the different conditions (groups) did not differ in terms of intrinsic motivation or competence as a factor that impacts needs satisfaction. Possible factors that contributed to this outcome is that the game did not provide enough challenge to the players, the feedback was insufficient in determining player performance and the game lacks visual and aural presentation of game elements and feedback to the player. We try to overcome these obstacles by designing a game that keeps the user constantly engaged by increasing challenge as the player progresses, provide visually appealing feedback, empowered social elements, encouraging self-empowering through performance graphs etc. 