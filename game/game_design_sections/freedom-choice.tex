\subsection{Freedom of choice}
%Autonomy - Freedom of choice - categories 
%training phase
During the first experiment where the non-gamified interface was used to perform annotations, many participants addressed the fact of not having control over the genre of the entities presented to them. Since the selection of the entities to be resolved was done in a random fashion, one of the participants was constantly getting entities that fell into the Arts category. As a consequence, the participant was feeling very insecure during the task because of being unfamiliar with most of the concepts being presented to him. We would like to stress out the importance of freedom of choice in this aspect. Being able to freely decide what category to play in, is an important factor for reinforcing autonomy and competence. As a result, the game gives the player the freedom of choice by providing several game categories to choose from. For players who like the aspect of surprise and chance, the game can make a random choice for the player if instructed to do so. 

Furthermore, the game provides a training phase for players who feel unprepared to take the actual task where the performance is recorded. According to Chamberlain et al. \cite{43} GWAP usually begin with a training phase so that players are able to practice their skills and also show that they have understood the instructions before they do the real task. However, in our case the onboarding stage takes care of explaining the complexity and instructions for performing actions in the game while the training phase allows the user to practice their typing skills. The training phase was also designed for the purpose of getting familiar with a new keyboard since the participants played the game from the experimenters laptop and therefore it was necessary to have a training phase. The game acknowledges the player for the existence of a training possibility in the game by pushing notification on the screen.