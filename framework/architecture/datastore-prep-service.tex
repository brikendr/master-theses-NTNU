\subsection{Data store \& data preparation service}
To assure the control and consistency of data being processed by the different microservices composing the framework, the \textit{Data-Store Service} was designed to be the only entry point to which the data could be manipulated. The service provides endpoints for accessing and manipulating the information residing on the database as a means of API calls and asynchronous message inquiries. It also serves as an information provider to the admin panel in the front-end application and also subscribes to different message routes such as: persisting named entities recognized from \ac{ner} microservice, persisting context clues and associating the generated candidate list to all registered entity mentions in the database. Besides subscribing to these message routes, the \textit{Data-Store Service} is the service endpoint that manages the complete asynchronous messaging infrastructure.

On the other hand, the \textit{Data-Preparation Service}, as the name implies, prepares the data for the AnnotateMe Interface as well as for the Fastype Game. Similar to the Data-Store Service, it has direct access to the database information with only one specific permission: reading (i.e. querying and retrieving information from the database). Therefore, for the sake of centralization and control of data, the Data-Preparation Service is considered as a read-only service with regards to the database access.

A unique feature implemented in the data preparation service is, as we like to call it, the \textit{disambiguation trigger}. This feature is responsible for resolving a specific entity mention (deciding which candidate represents the correct link for the target entity) when enough annotation data from the human annotators are accumulated. Since the most usual use-case scenario includes non-expert human annotators performing the validation process by either using the AnnotateMe Interface or the Fastype game, assuring quality of annotations is reached through redundancy. The level of redundancy maintained in the data preparation service is based on constraints proposed by Snow et al. \cite{32}. They conducted an experiment where they evaluated the quality of non-expert annotators in comparison with expert annotators. Their results indicate that on average it requires 4 independent non-expert annotations to achieve the equivalent ITA of a single expert annotator. Therefore, the disambiguation trigger is triggered when 4 independent annotations (having the same candidate as the selected option) have been accumulated for a specific entity mention. After an entity has been resolved, it will no longer show up on the interface for validation. The resolving step is done by the \textit{Data-Preparation Service} which initiates a REST API call to the \textit{Data-Store Service} in order to update the information on the databse.

%Order effects
A final element that is taken care by the \textit{Data-Preparation Service} is the ordering of candidates presented to human annotators. In a study conducted by Duarte et al. \cite{34}, they argue that in search engines, web users expect the best answer to be in the first or second position. This type of expectation represents potential bias on the results assessing the behaviour of annotators. After performing some user studies, they conclude that search result selection behaviour is influenced by ranking, with users showing tendency to select higher ranks without exploring other alternatives. To avoid having this situation in our experiment, we encourage users to explore all the candidates presented to them before making a decision. The encouragement is done by providing short descriptions for each candidate on the interface. Additionally, we avoid the chance of making users form assumptions about potential candidates being ranked higher in a list by completely randomizing the process of candidate positioning. Random positioning of alternatives instead of ranking has proven to be much more effective in encouraging users to explore all available alternatives instead of making blind decisions \cite{34}.
