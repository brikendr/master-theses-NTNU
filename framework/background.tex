\section{Background}
\label{framework:background}
The Named Entity Disambiguation process is usually composed of two different modules: named entity recognition and candidate generation which also does the entity linking and disambiguation. However, in this research study we extend the number of modules composing the framework in order to provide more information to the human annotators for conducting the disambiguation process. The additional modules implemented in our framework consist of a module for extracting contextual clues around the named entities called \textit{Context-Clue Generation}, a module for generating \textit{topical clues} describing the general topic of the document in which the entity resides, a \textit{Data Preparation} module which prepares the data required by the \textit{AnnotateMe Interface} which in turn uses human annotators to validate the generated links by the \textit{Candidate Generation} module. For a better understanding, lets take an example of a text fragment and explain what the different modules produce as an outcome.

\begin{quote}
 "While Apple is an electronics company, Mango is a clothing one and Orange is a communication one." - excerpt taken from KORE50 Dataset
\end{quote}

The text fragment above consists of surface forms (named entities) with a very ambiguous nature that is genuinely hard for automatic annotators to disambiguate. The responsibility of the entity recognition module is to identify the corresponding named entities in the text fragment. These entities represent real-world objects and usually fall into one of the following categories: Organization, Location and People. Running the named entity recognition module on the above text fragment would result in the identification of the following entities: Apple, Mango and Orange (all three being identified as Organizations). During this step, a commonly known technique for identifying the entities is using classifiers \cite{13}. Our framework utilizes the Stanford Named Entity Recognizer (Stanford \ac{ner}) for this purpose which is known as a CRFClassifier \cite{standfordNER}. According to Finkel et al. \cite{standfordNER} the recognizer uses a general implementation of a linear chain Conditional Random Field (CRF) sequence models. These models are trained using labeled data and are generally classified as supervised approaches. Our framework uses SanfordNER\footnote{Stanford NER Package \url{https://nlp.stanford.edu/software/CRF-NER.shtml}} with a standard CRFClassifier for the English language trained with features for 3 classes in particular (Organization, Location and People). However, the classifier can be extended for recognizing additional classes and in other languages as well, but this problem is out of the scope of this research study and therefore we use the standard classifier. It is important to note that StanfordNER is one of the tools used by the framework for recognizing entities within text fragments. The complete process will be explained later in Section \ref{framework:architecture}. 

After the named entities have been identified by the recognition module, the framework proceeds by extracting contextual clues for each individual entity as well as document keywords. The importance of appropriately formulating the surrounding context in which the entity mentions occur has been explained in Section \ref{background:defininf_context}. Therefore, the Context-Clue Extraction and Topic-Keyword Extraction modules represent the crucial part of the information presented to the human annotator in order for them to make an accurate disambiguation. From the text fragment above, an accurate context clue for disambiguating the named entity \textit{Apple} would be \textit{electronics company}. From an annotation point of view, the extracted contextual clue such as \textit{electronics company} for entity \textit{Apple} would provide sufficient information for the human annotator to decide whether the entity \textit{Apple} refers to the plant or Apple Inc. Topic keywords on the other hand, keywords that represent the general context of the document, are more useful when processing larger text chunks rather than short sentences. In the text fragment above, the most appropriate and useful topic keywords would be: Apple, Orange, Mango and company. 

The final step for completely resolving the text fragment example provided above is by generating candidates for each identified entities from an \ac{lod} knowledge base and disambiguating the entities by picking one candidate among many which best represents its meaning within the defined context. The former is done in an automatic fashion by utilizing an automatic annotator whereas the later is done by asking human annotators to pick the correct candidate for each entity. If we take the entity "Apple" as an example, the candidate generation module generates the following candidates: 
\begin{itemize}
    \item Apple (Plant, Species, Eukaryot)
    \item Apple Records (Company, RecordLabel)
    \item Apple Inc. (Organization, Company)
    \item Apple II (Computer)
\end{itemize}

A weighted score is correspondingly assigned to each candidate by the automatic annotator. This score represents the level of confidence which is a numerical value from 0 to 1. The candidate which has the greatest confidence score is the best representative for the entity Apple as judged by the automatic annotator. There are cases in which the automatic annotator fails to correctly disambiguate the entity by picking the wrong candidate, or not providing the right candidate in the list at all. It is the responsibility of a human annotator to decide on the correct candidate from the list (if listed) based on the contextual information provided as short clues.

The aim of this section was to establish a general understanding of the different modules composing the framework and their corresponding responsibilities on building the foundations for an effective and qualitative named entity disambiguation work-flow. Section \ref{framework:architecture} will analyze and explain the underlying technical details for each module. 
\newpage