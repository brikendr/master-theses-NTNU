\section{Game With A Purpose (GWAP)}
\label{thb:gamedesign}
%Why is gamification applied in our work
\if The advancement of technology has brought us many innovations and systems which contribute to facilitating everyday life and automating boring human workforce. However, the level of advancement has yet to reach the point where the human expertise is not required. Until then, many researchers are trying to explore ways in which particular boring tasks which are usually carried out by humans can become more interesting and exciting to do. \fi Gamification represents the idea of using game design principles for transforming a system which was originally not created as a fun activity into an engaging and interactive game with a purpose \cite{vonahn, 47}. In this study we elaborate on the potential of applying game design principles into the task of named entity disambiguation in order to achieve large-scale annotation data that can be used either as training data for supervised NLP algorithms or as a tool for enriching web documents with semantic meta-data.

%What is gamification - definitions
The proposed gamification approach of this research study and the field in which this technique is being applied, in literature is referred to as Games With A Purpose (\ac{gwap}) \cite{vonahn}.\ac{gwap} is one of the many approaches of gamification. The concept of gamification, as seen by researchers, involves applying elements of \textit{gamefulness, gameful interaction and gameful design} with a specific intention in mind. According to Seaborn et al. \cite{47}, \textit{gamefulness} refers to the lived experience, \textit{gameful interaction} refers to the object, tools and contexts that bring the feeling of gamefulness while \textit{gameful design} refers to the practice of creating a gameful experience. On the other hand, the aim of \ac{gwap} is to entertain players while they complete tasks that the system does not know, for most of the part, the correct answer. Usually, a well design \ac{gwap} harvests the knowledge of their players and acquires the solution to the underlying problems as a byproduct of players playing and interacting with the game \cite{vonahn}. Providing appropriate feedback at appropriate times during the gameplay for the players to feel engaged while using their knowledge and experience to solve \ac{nlp} problems presents a major challenge. Understanding the motivation of players in this scenario is key to the success of a \ac{gwap} \cite{43}.

%Why do we base our gamifying process on theoritical models 
Being able to guarantee, to some extent, that players will be engaged and motivated to play while interacting with the game, certain psychological needs have to be fulfilled so that players have the feeling of being immersively away from the real world and fully concentrated on the game. The answer to this question lies on theoretical cognitive theories such as the widely used Self-Determination Theory (\ac{sdt}) and its respective sub-theories \cite{std_and_games}. The process of designing the game for this particular research problem has been completely based on theoretical foundations and psychological theories for motivation (i.e. \ac{sdt}) in order to reach a state where players experience the feelings of being entertained. 

%What is SDT and CET, Intrinsic, extrinsic motivators and competance 
\ac{sdt}, is a macro theory of human motivation that is essentially concerned with the potential for social contexts to provide satisfying experience. In \ac{sdt}, the importance of competence (i.e. outcome control), autonomy (i.e. agency) and relatedness (i.e. connecting with others) are emphasized as the main factors to intrinsic motivation. Intrinsic motivation denotes the pursuit of an action because it is inherently enjoyable or interesting. In contrast, extrinsic motivation is defined as doing something due to a separable outcome, such as pressure or "extrinsic rewards" in the form of payment incentives or verbal feedback (ex. praise). Competence on the other hand, signifies the perceived extent of ones own actions as a cause of desired consequences and, as a psychological factor, is increased when the corresponding action is met with direct and positive feedback. It must be noted here that feelings of competence will not increase intrinsic motivation unless accompanied with a sense of autonomy. To affect feelings of autonomy, people must experience their actions and behaviour as self-determined rather than controlled by the system or an outside source. In support towards this concept, Cognitive Evaluation Theory (\ac{cet}), a sub-theory of \ac{sdt}, suggests that activities/actions foster greater intrinsic motivation when they provide goal-oriented tasks and an effort-full challenge. \cite{43, 49,std_and_games}