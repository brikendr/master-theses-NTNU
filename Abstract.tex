\addcontentsline{toc}{chapter}{Abstract}
\chapter*{Abstract}

%What is it about not what it tells
%=======Improvment of named entity disambiguation systems and further advancing the Semantic Web y enriching unstructured text documents with semantic metadata and NLP problems by providing meas of generating training data for supoervised approaches in order to improve their performance and accuracy.
%=====Gamification of systems that were not initially designed for gaming contexts
%=====intrinsically motivate human annotators y leveraging and applying game elements that affect psychologigacl needs for satisfaction.
%======Support non-expert users to perform quality data generation comparted to expert users 
%======Among the very few studies in gamification/gwap that bases its work on empirical research and stadard of practice for design and implementation
%It should contain the essential qualities of this works' composition
%The abstract must be able for substituting the whole thesis when there is inssuficient time and space for the full text


\textit{The content generated on the web originates from diverse sources with the main purpose of serving updated information to the  Internet user. Every piece of information generated is valuable and must be easily traced by modern search engines. Semantic meta-data as a mechanism for providing meaning to the generated content is the de-facto requirement for improving search accuracy and facilitating information discovery on the web. This research represents an attempt for advancing the field of semantic web in terms of providing an approach for generating semantic information to the substantial number of unstructured documents available on the web. The disambiguation of recognized named entities within the content of these documents represents the problem elaborated in this work which contributes to semantically linked web of data. The implementation of a generic and scalable gamified named entity disambiguation framework demonstrating the capabilities of non-expert users in generating large-scale annotation data represents the main qualities composing this research study. This study specifically focuses on benefiting from gamification as a powerful and prominent approach for leveraging human computation. A gamified named entity disambiguation framework which intrinsically motivates non-expert users by applying game elements that affect psychological needs for satisfaction ultimately supports the generation of large-scale and qualitative annotation data. It provides confident results and analysis as it is among the few research studies in gamification that bases its work on empirical research and standard practice for design and implementation.}

\hypersetup{pageanchor=false}