\chapter{Methodology}
\label{chap:methodology}

- General approach used to gather relevant literature, implement framework, prepare data, conduct experiments and anaylzing the data (Qualitative and Quantitative research approaches)

\section{Evaluation Dataset}
\subsection{MSNBC Dataset}
%Characteristics of the dataste: Type (Formal, informal - short vs long), nature, timespan, nr of documents etc
\subsection{KORE50 Dataset}
'
\subsection{Spotlight Dataset}

\subsection{GERBIL Benchmarking Framework}

- GERBIL provides comparable results to tool developers so as to allow them to easily discover the strengths and weaknesses of their implementations with respect to the state of the art. It is an open source and extensible framework that allows evaluating tools against (currently) 9 different annotators on 11 different datasets within 6 different experiment types. 
- GERBIL significantly improves the time-to-evaluation by offering means to benchmark and compare against other annotators respectively datasets within the same effort frame previously required to evaluate on a single dataset.

- Usually entity linking systems are assessed in terms of evaluation measures such as [12]:
\begin{itemize}
    \item Precision
    \item Recall
    \item F-measure
    \item Accuracy
\end{itemize}
- However, end-user studies are evaluated by rating the applications usability, user interaction, fun factor and by measuring time spent on engaging with the interface. 


\subsection{Dataset Statistics}
    - Nr of entities: Person, Location and Organization
    - Percentage of Linkable Entities
    - Percentage of NIL entity mentions (i.e. entity mentions that do not have a corresponding counterpart in the KB)
    - Percentage of ambiguous words
    
\section{Participants}
    - Statistics about the participants like:
        - Age, gender, education, experience with annotating tasks

- Task Experts: are domain experts who conceptually understand the task of annotating text and have insights into the characteristics of semantic annotation [15].
\section{Questionnaires}
    - Describe the questionnaires that will be used in the user study (pre and post-questionnaires)